\documentclass[11pt]{extarticle} 
\usepackage{graphicx} % Required for inserting images
\usepackage{amsfonts}
\usepackage{enumitem}
\usepackage[hidelinks]{hyperref}
\usepackage{graphicx}
\usepackage{textcomp}
\usepackage{amsmath}
\usepackage{multicol}
\usepackage{mathabx}
\usepackage[bottom=2cm, right=2.5cm, left=2.5cm, top=2cm, headheight=16pt]{geometry}
\usepackage{amssymb}
\usepackage{amsthm}
\usepackage{physics}
\usepackage{cancel}
\usepackage{mathtools}
\usepackage{array}
\usepackage{tikz}
\usepackage{MnSymbol}
\def\checkmark{\tikz\fill[scale=0.4](0,.35) -- (.25,0) -- (1,.7) -- (.25,.15) -- cycle;} 
\makeatletter
\newcases{crcases}{\quad}{%
  \hfil$\m@th\displaystyle{##}$\hfil}{\hfil$\m@th\displaystyle{##}$}{\lbrace}{.}
\makeatother

\title{[American Politics] Final Paper}
\author{Connor Li (connorsean.li$@$sciencespo.fr)\\
\\
Professor: Anne-Laure Beaussier}

\date{December 4, 2024}

\begin{document}

\maketitle

\tableofcontents



\pagebreak

\section{Introduction}
Education is often heralded as the cornerstone of the American Dream, a pathway to opportunity and upward mobility that embodies the nation’s ideals. Yet in the United States, the governance of education reflects the complexities of federalism—a system where authority is divided among federal, state, and local governments. This decentralized framework allows for tailored solutions to local needs, but it also perpetuates disparities in quality, access, and outcomes. These challenges go beyond funding inequities to include inconsistencies in curriculum standards, accountability systems, and the ability to respond effectively to crises.
\subsection{Defining Federalism in Education}
Federalism, a defining principle of American governance, assigns distinct roles to different levels of government. In education, states and local governments hold primary responsibility, overseeing curriculum, teacher certification, and funding mechanisms. The federal government supplements these efforts by providing targeted funding and setting broad accountability measures. However, the balance of power between these levels has shifted significantly over time:
\begin{quote}
    ``The system of educational governance facilitates a division of power and control among the three planes of government, namely, federal, state, and local. Although the federal government has expanded its involvement in educational policy since the 1960s, public education remains the primary responsibility of state and local government" (Federalism Project).
\end{quote}
This layered system, while necessary to address the nation’s diverse needs, often results in uneven outcomes, particularly in underserved communities.
\subsection{Historical Context}
Historically, education governance in the United States was rooted in local control. Early federal involvement, such as the Northwest Ordinance of 1787, encouraged the establishment of schools but left operational authority to states and municipalities (The Conversation). This decentralized model persisted until the mid-20th century, when legal and social developments underscored the need for greater federal engagement. Landmark decisions like Brown v. Board of Education (1954) and the Civil Rights Act of 1964 highlighted the federal government’s responsibility to address inequities in access and desegregation (Harvard Graduate School of Education).\\
\\
These changes were followed by legislative initiatives such as the Elementary and Secondary Education Act (ESEA) of 1965, which tied federal funding to efforts to reduce educational disparities. Reauthorizations like the No Child Left Behind Act (2001) and the Every Student Succeeds Act (2015) illustrate how federal education policy has alternated between increased oversight and a return to local control (Columbia Law Review). These shifts reflect broader tensions within federalism, as states seek to maintain autonomy while meeting federally mandated standards.
\subsection{Contemporary Relevance}
Federalism’s role in education remains highly relevant today. The COVID-19 pandemic exposed glaring disparities in how states and localities responded to crises, revealing fragmentation in governance and resource allocation. For example, federal relief funding under the CARES Act varied widely in its implementation and effectiveness, exacerbating existing inequities in digital access and infrastructure. Simultaneously, debates over curriculum content, such as the teaching of critical race theory, illustrate how federalism amplifies ideological divides, with states adopting widely divergent approaches to education policy.\\
\\
These contemporary challenges underscore the urgent need to examine how federalism can adapt to address persistent inequities while maintaining its strengths, such as localized decision-making and innovation. Understanding these dynamics is essential to ensuring that education fulfills its promise as a pathway to opportunity for all students.
\subsection{Central Focus}
This paper examines the role of federalism in shaping public education governance in the United States, focusing on how the decentralized structure of authority influences disparities in funding, resource allocation, and accountability. Through an exploration of key case studies—including the No Child Left Behind Act and the Every Student Succeeds Act—this paper demonstrates how the interplay between federal, state, and local governments has both driven innovation and perpetuated systemic inequities.\\
\\
By grounding the analysis in historical and contemporary contexts, this study highlights the broader tensions and contradictions inherent in federalism. It explores the ways in which federal oversight has been used to address inequities, as well as how localized autonomy has created challenges in achieving national standards of equity and access. The case studies serve to illustrate these dynamics in action, showing the strengths and limitations of federalism as a model for governing education.\\
\\
Ultimately, this paper seeks to illuminate how federalism continues to shape the U.S. education system, providing insight into its enduring impact on policy, governance, and equity. The analysis also considers potential reforms, emphasizing how the strengths of federalism can be harnessed while mitigating its weaknesses to create a more equitable and effective educational framework.




\section{}






\end{document}
