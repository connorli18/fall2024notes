\documentclass[11pt]{extarticle} 
\usepackage{graphicx} % Required for inserting images
\usepackage{amsfonts}
\usepackage{enumitem}
\usepackage[hidelinks]{hyperref}
\usepackage{graphicx}
\usepackage{textcomp}
\usepackage{amsmath}
\usepackage{multicol}
\usepackage{mathabx}
\usepackage[bottom=2cm, right=2.5cm, left=2.5cm, top=2cm, headheight=16pt]{geometry}
\usepackage{amssymb}
\usepackage{amsthm}
\usepackage{physics}
\usepackage{cancel}
\usepackage{mathtools}
\usepackage{array}
\usepackage{tikz}
\usepackage{MnSymbol}
\usepackage{hanging}
\usepackage[backend=biber,style=apa]{biblatex}
\usepackage{url}

\addbibresource{references.bib}


\def\checkmark{\tikz\fill[scale=0.4](0,.35) -- (.25,0) -- (1,.7) -- (.25,.15) -- cycle;} 
\makeatletter
\newcases{crcases}{\quad}{%
  \hfil$\m@th\displaystyle{##}$\hfil}{\hfil$\m@th\displaystyle{##}$}{\lbrace}{.}
\makeatother

\title{[American Politics] Final Paper}
\author{Connor Li (connorsean.li$@$sciencespo.fr)\\
\\
Professor: Anne-Laure Beaussier}

\date{December 4, 2024}

\begin{document}

\maketitle

\tableofcontents



\pagebreak

\section{Introduction}
Education is often heralded as the cornerstone of the American Dream, a pathway to opportunity and upward mobility that embodies the nation’s ideals. Yet in the United States, the governance of education reflects the complexities of federalism—a system that divides authority among federal, state, and local governments. This decentralized framework allows for tailored solutions to local needs but perpetuates disparities in quality, access, and outcomes, including funding inequities, inconsistent standards, and uneven crisis management.\\
\\
Understanding these challenges requires an exploration of the unique governance structure that shapes the education system. Federalism, as a defining principle of American governance, balances state and local control over curriculum, teacher certification, and funding with federal efforts to provide resources and ensure accountability. This balance has shifted over time, as federal involvement expanded to address national priorities such as equity, accountability, and innovation.

\subsection{Defining Federalism in Education}
Federalism assigns distinct roles to different levels of government, with state and local authorities traditionally holding primary responsibility for public education. States determine teacher certification, graduation standards, and curriculum guidelines, while local governments oversee school district operations and funding mechanisms, often through property taxes (Wong). The federal government supplements these efforts by providing conditional funding and setting broad accountability measures.\\
\\
This expanded role is evident in initiatives like Title I of the Elementary and Secondary Education Act (ESEA), which allocates federal resources to schools serving high concentrations of low-income students (Pelsue). Similarly, the Individuals with Disabilities Education Act (IDEA) ensures that states provide free and appropriate public education for students with disabilities (Pelsue). Federal laws like No Child Left Behind (NCLB) and its successor, the Every Student Succeeds Act (ESSA), introduced accountability frameworks that require standardized testing and performance metrics to close achievement gaps. While these measures promote equity, they are often criticized for imposing burdensome compliance requirements on schools (Wong).

\subsection{A Brief Historical Context}
The governance of education in the United States has historically been rooted in local control, reflecting the nation's founding principles. The Northwest Ordinance of 1787 serves as an early example of federal encouragement for education, emphasizing the importance of schooling while leaving operational authority to states and municipalities (Hornbeck). This decentralized model was rooted in the belief that local communities were best equipped to address their educational needs.\\
\\
For much of the 19th and early 20th centuries, this approach remained largely unchallenged. However, mid-20th century developments exposed its limitations in addressing widespread inequities. The Supreme Court’s decision in \textit{Brown v. Board of Education} (1954) underscored the federal government’s responsibility to combat racial segregation, declaring that “separate but equal” education was inherently unequal (Wong). Legislative milestones like the Civil Rights Act of 1964 further positioned the federal government as a key player in promoting educational equity (Hornbeck).\\
\\
The Elementary and Secondary Education Act (ESEA) of 1965 marked the first substantial federal investment in public education (Kosar). By tying funding to efforts aimed at reducing poverty-related disparities, the ESEA demonstrated how financial incentives could influence state and local policies. Subsequent reauthorizations, such as No Child Left Behind (2001) and the Every Student Succeeds Act (2015), illustrate the ongoing tension between federal oversight and state autonomy. These policies highlight the cyclical nature of governance, as shifts between federal mandates and local control reflect broader debates about equity, innovation, and accountability (Kosar).

\subsection{Contemporary Relevance}
Federalism continues to play a pivotal role in addressing modern educational challenges. The COVID-19 pandemic starkly revealed disparities in how states and localities managed disruptions, highlighting the fragmentation of governance and inequities in resources. While the federal government allocated \$13.5 billion in emergency relief to K–12 schools through the CARES Act, the distribution and utilization of these funds varied widely across states (Walsh, Moynihan, Yin). Wealthier districts were often able to transition to remote learning quickly, while underfunded schools struggled with inadequate access to technology and broadband (Walsh).\\
\\
Ideological divides across states further complicate the landscape. Debates over curriculum content, such as the teaching of critical race theory and LGBTQ+ representation, showcase how federalism allows for widely differing policies. For instance, states like Florida have passed laws restricting certain topics, while states like California have expanded inclusive curricula (Wong). These divergent approaches reflect the broader cultural and political divides that federalism can amplify.\\
\\
Beyond crises and cultural debates, federalism’s impact is evident in ongoing challenges related to equity and innovation. Programs like Title I and IDEA aim to address resource disparities but often highlight the limitations of federal mandates in achieving uniform outcomes. At the same time, local autonomy has enabled states and districts to experiment with innovative approaches, such as competency-based education and expanded vocational training. However, this flexibility can also result in a “race to the bottom,” where reduced standards or funding exacerbate inequalities.

\subsection{Central Focus}
This paper examines how federalism shapes public education governance in the United States, focusing on its influence on disparities in funding, resource allocation, and accountability. Through an exploration of key case studies—including No Child Left Behind and the Every Student Succeeds Act—this paper illustrates how the interplay between federal, state, and local governments has driven both innovation and inequity.\\
\\
By grounding the analysis in historical and contemporary contexts, this study highlights the broader tensions inherent in federalism. It explores how federal oversight has been used to address inequities, while also considering how localized autonomy can hinder the achievement of national standards. These case studies serve to demonstrate the strengths and limitations of federalism, providing insight into its enduring impact on policy, governance, and equity.
Ultimately, this paper seeks to illuminate how federalism continues to shape the U.S. education system and considers potential reforms to harness its strengths while addressing its weaknesses.




\pagebreak
\section{Relevant Policy Case Studies}
\subsection{No Child Left Behind Act (NCLB)}
The No Child Left Behind Act (NCLB) of 2001 is a pivotal example of the tension between federal goals and state autonomy, highlighting both the potential and limitations of federalism in addressing systemic disparities in public education.
\subsubsection{Increased Federal Oversight}
Enacted as a reauthorization of the Elementary and Secondary Education Act (ESEA), NCLB expanded federal oversight, introducing stringent accountability measures tied to federal funding by requiring states to meet detailed educational standards. 
The law mandated annual standardized testing in reading and math for students in grades 3 through 8, using these results to assess school performance and close achievement gaps among student subgroups, including racial minorities, low-income students, and English language learners (McGuinn). Schools were required to make "adequate yearly progress" (AYP) toward proficiency, with a 12-year timeline for achieving 100\% student proficiency (McGuinn). Failure to meet AYP benchmarks triggered federal interventions, including restructuring schools and providing alternative educational options for students.\\
\\
This increased oversight marked a dramatic departure from the historical federalist model of education governance. By tying federal Title I funding to compliance with its mandates, NCLB expanded the federal government’s influence over a traditionally state-dominated domain. Specifically, it was noted that NCLB ``raises fundamental issues about who controls education,” shifting significant authority to the federal government to define failing schools and mandate specific remedies (Sunderman, Kim).
\begin{quote}
  ``The aggressive implementation approach of the Bush administration’s Department of Education succeeded in getting states to comply with federal mandates and intervene to a greater extent than ever before in districts with low-performing schools'' (McGuinn).
\end{quote}
While this federal direction aimed to improve transparency and accountability, it also created tensions with state and local governments, which often struggled to meet the new demands.
\subsubsection{Funding and Accountability}
While NCLB significantly increased federal education spending—Title I funding alone grew by approximately 50\% in its first year—these additional resources were often insufficient to meet the law’s demands. 
States and districts faced substantial costs to implement the testing and accountability systems required under NCLB, exacerbating existing fiscal pressures. 
For example, state budget shortfalls during the early 2000s made it difficult for many states to allocate sufficient resources to comply with federal requirements, leading to delays and uneven implementation across the country (Sunderman, Kim).\\
\\
Moreover, the law’s focus on accountability created incentives that often undermined its broader educational goals. Many schools and districts engaged in ``teaching to the test,” narrowing the curriculum to prioritize subjects included in standardized assessments, such as math and reading, at the expense of other areas like science, social studies, and the arts. Some states lowered proficiency standards to inflate test scores and avoid federal sanctions, a phenomenon critics described as “gaming the system” (Dee, Jacobs).
These behaviors highlighted the disconnect between the federal government’s policy objectives and the realities of local implementation, raising questions about the efficacy of NCLB’s approach to accountability.
\subsubsection{Federal-State Tensions}
The heightened federal involvement under NCLB also exacerbated tensions between federal and state governments, as well as within states themselves. By imposing strict timelines and detailed compliance requirements, the law challenged traditional notions of state and local autonomy in education. States that failed to meet NCLB mandates risked losing federal Title I funds, a critical source of support for schools serving disadvantaged populations (Sunderman, Kim). 
This enforcement mechanism was intended to ensure compliance but often alienated state and local officials, who viewed it as an overreach of federal authority (Sunderman, Kim).\\
\\
Resistance to NCLB grew as its implementation progressed. By 2004, several states, including Utah and Virginia, had passed resolutions opposing the law or seeking relief from its requirements. These resolutions reflected bipartisan dissatisfaction with the burdens NCLB placed on state and local governments, as well as concerns about its impact on local priorities. For example, in Chicago, district officials criticized the law’s transfer provisions, arguing that they undermined community goals by forcing schools to accept students from outside their neighborhoods, potentially lowering overall performance metrics (Dee, Jacobs).\\
\\
Political opposition to NCLB extended beyond state governments to include educators and district administrators, who cited the law’s bureaucratic demands and punitive approach as major barriers to meaningful reform. Critics argued that NCLB prioritized compliance over capacity-building, leaving schools and districts without the professional expertise or resources needed to achieve its ambitious goals (McGuinn). This disconnect between federal policy and local practice underscored the challenges of implementing national education reforms within a federalist system (Sunderman, Kim).
\subsubsection{Effectivness and Limitations}
Among experts, NCLB’s effectiveness as a federal intervention in education remains hotly debated. On one hand, the law succeeded in increasing transparency, focusing attention on chronically underperforming schools and highlighting achievement gaps among disadvantaged student populations. It also spurred the development of statewide data systems and fostered a greater emphasis on accountability of educational reform.\\
\\
However, these successes were tempered by significant challenges. Many states and districts lacked the resources or expertise needed to implement NCLB’s requirements effectively. The law’s punitive approach, combined with its reliance on standardized testing, often undermined broader educational goals, such as fostering critical thinking and creativity. Specifically, NCLB’s strict timelines and rigid accountability measures “did not generate as much meaningful school improvement or progress in closing student-achievement gaps as was originally hoped” (Sunderman, Kim).\\
\\
The law’s federalism implications are equally as complex. By expanding the federal role in education, NCLB demonstrated the potential of federalism to drive national priorities and enforce accountability. Yet, it also revealed the limitations of a top-down approach, as states and localities struggled to reconcile federal mandates with their own needs and capacities. This tension reflects a broader challenge within federalism: balancing national equity goals with the flexibility and innovation afforded by local autonomy.





\subsection{Every Student Succeeds Act (ESSA)}
The Every Student Succeeds Act (ESSA) of 2015 represents a significant recalibration of the federal-state relationship in education, following the expansive federal oversight introduced by the No Child Left Behind Act (NCLB). ESSA sought to shift authority back to states and local districts, allowing them greater flexibility in setting academic standards, implementing accountability measures, and allocating resources. This shift not only marked a departure from the prescriptive nature of NCLB but also reignited debates about the effectiveness of federalism in achieving educational equity.
\subsubsection{A Return to State Control}
As mentioned above, ESSA was passed in response to growing resistance to NCLB’s federal mandates, which many viewed as overreaching and overly rigid. 
Under ESSA, states regained the ability to develop their own academic standards and accountability systems, with minimal federal interference. 
For instance, the law requires states to identify and intervene in the bottom 5\% of underperforming schools and high schools with graduation rates below 67\%, but it leaves the design and implementation of these interventions to the discretion of state governments (Heise). 
This “devolution of power” was hailed by some policymakers as a restoration of state autonomy---“a significant devolution of power” in K-12 education (Heise).\\
\\
In contrast to NCLB’s prescriptive measures, ESSA offers states the flexibility to determine how to meet federal goals. 
For example, states can choose non-academic factors, such as school climate or student engagement, to include in their accountability systems (Saultz, Fusarelli, McEachin). They are also allowed to design teacher evaluation frameworks without federal constraints, a departure from the Obama administration’s use of competitive grants and waivers to promote specific policies like tying teacher evaluations to student test scores (Saultz, Fusarelli, McEachin).
\begin{quote}
  “The Every Student Succeeds Act permits states to develop, test, and measure academic metrics and standards. Aside from a requirement for standards to be ‘challenging,’ ESSA now delegates to the states the task of developing academic standards” (Heise).
\end{quote}
This flexibility reflects a broader philosophy of ESSA: to empower states as laboratories of experimentation while minimizing federal intrusion. However, this approach has also raised concerns about the consistency and rigor of state policies, particularly in addressing disparities in educational outcomes.
\subsubsection{Inequity and Accountability}
While ESSA grants states greater authority, critics argue that this decentralization undermines efforts to ensure equity and accountability. One of ESSA’s key weaknesses is its lack of enforcement mechanisms. Unlike NCLB, which imposed federal sanctions on schools failing to meet annual progress targets, ESSA largely relies on states to hold themselves accountable. This hands-off approach has drawn criticism from civil rights organizations and education advocates, who worry that states may neglect disadvantaged student populations.
\begin{quote}
  ``ESSA limits the ability of the federal government to impose consequences on states and districts for student performance, causing some to worry that ESSA may hinder educational equity” (Walsh, Moynihan, Yin). 
\end{quote}
Similarly, Derek Black warns that ESSA’s emphasis on state discretion could exacerbate existing inequalities, as states are not required to align their policies with rigorous equity standards'' (Sundquist). The potential for inequities is particularly acute in resource allocation. While ESSA requires states to submit plans for supporting disadvantaged subgroups, many states have failed to meet these equity requirements. 
For example, a study found that low-income schools are more likely to employ inexperienced or underqualified teachers, a problem that ESSA’s “bare minimum” certification standards do little to address (Walsh, Moynihan, Yin). These disparities reflect broader issues in education federalism, where decentralized governance can lead to uneven outcomes across states and districts.
\begin{quote}
“Critics of ESSA contend that it moves education in a direction that was unthinkable just a few short years ago: no definite equity provisions, no demands for specific student achievement, and no enforcement mechanism to prompt states to consistently pursue equity or achievement” (Heiss).
\end{quote}
\subsubsection{Federalism's Implications and Challenges}
Proponents of ESSA argue that its decentralized approach fosters innovation and allows states to tailor policies to local needs. 
By giving states greater flexibility, ESSA encourages experimentation and collaboration between state and local governments. 
For instance, some states have used ESSA to pilot initiatives aimed at improving school climate, integrating technology into classrooms, or addressing mental health needs (Sundquist). These efforts highlight the potential of federalism to drive localized solutions to educational challenges.\\
\\
However, this flexibility also comes with risks. ESSA’s reliance on states to self-regulate has led to inconsistent policy implementation and a lack of uniformity in addressing systemic issues. A review of state ESSA plans revealed that fewer than half adopted robust definitions of equity, while many continued to rely on vague or inconsistent metrics (Sundquist). This variability underscores the limitations of decentralized governance in addressing nationwide challenges, such as racial and socioeconomic disparities in education.\\
\\
COVID-19 further exposed the weaknesses of ESSA’s decentralized framework. As states grappled with the transition to remote learning, many struggled to meet ESSA’s accountability requirements. Waivers were granted to states unable to conduct standardized testing, raising questions about the long-term viability of ESSA’s accountability provisions in times of crisis (Walsh, Moynihan, Yin). This experience illustrates the fragility of a system that relies heavily on state and local capacities, particularly during periods of heightened demand.
\subsection{COVID-19: Federalism in Crises}
On a larger scale apart from out discussion of ESSA, the COVID-19 pandemic exposed significant challenges in the federalist governance of U.S. education. As schools abruptly transitioned to remote learning, the decentralized structure left states and local districts largely responsible for implementing public health measures, distributing resources, and addressing disparities in access to technology. As a result, this fragmented response resulted in wide variations in outcomes.\\
\\
Federal funding under the Coronavirus Aid, Relief, and Economic Security (CARES) Act aimed to mitigate these disparities by allocating \$13.2 billion to the Elementary and Secondary School Emergency Relief (ESSER) Fund (García, Weiss). However, the lack of cohesive federal guidelines led to uneven distribution and use of these funds. Wealthier districts with more robust administrative capacity often accessed funding more effectively, while under-resourced schools struggled. Critics noted that the digital divide, a long-standing issue in education, widened during the pandemic, with low-income students and those in rural areas disproportionately affected.\\
\\
The crisis also highlighted inefficiencies in balancing federal oversight with local autonomy. States retained broad discretion over fund allocation, leading to inconsistent prioritization of reopening plans versus equity initiatives. As one analysis observed:
\begin{quote}
``The response to COVID-19 demonstrated how education federalism amplifies disparities during crises, as states with stronger fiscal and technological capacity were able to weather the storm more effectively than their less-resourced counterparts” (Pears, Sydnor).
\end{quote}
So while federal funding was critical, its fragmented implementation exacerbated existing educational and economic disparities. Moving forward, the experience of the COVID-19 crisis highlights the need for more structured federal-state collaboration to ensure equitable and effective responses to future disruptions.










\section{Challenges in Educational Federalism}
As shown through the case studies, federalism---while intended to provide a balance between national oversight and localized governance---has created an inefficient system in public education. This structure is marked by overlapping roles, inequitable funding mechanisms, and a misalignment between national goals and local implementation. 
\subsection{Fragmentated Authority and Accountability Gaps}
Specifically, the decentralized structure of education governance results in fragmented authority, where overlapping responsibilities lead to inconsistent policies and inefficiencies. This division is most apparent in the varying quality of education across states and districts. Under the model of dual federalism, states and localities are granted broad discretion in determining educational standards, curricula, and funding allocations, often to the detriment of national equity goals.
\begin{quote}
``The model of federalism employed in K–12 education, dual federalism, gives substantial deference to state-level legislative, executive, and judicial decisions—especially regarding educational quality and school funding. Under this approach, variation among states is not only expected but, in fact, is encouraged. The underbelly of this deference and variation is that the proverbial floor of educational quality is uneven between states” (Bowman).
\end{quote}
This fragmented governance model also hinders the effective monitoring and enforcement of federal education programs. For example, Title I of the Elementary and Secondary Education Act (ESEA) was designed to provide additional resources to low-income students. However, ambiguities in guidelines and weak enforcement mechanisms often result in these funds being misallocated.
\begin{quote}
``The problems with misappropriation of federal aid began soon after Title I [...] money given to school districts was not actually spent on the education of low-income children” (Hills).
\end{quote}
Instead of addressing disparities, these funds are frequently redirected to politically influential constituencies, highlighting a fundamental misalignment between federal intent and local implementation (Harvard).
Furthermore, federal education programs often prioritize compliance over meaningful outcomes, further entrenching systemic inequities. As we saw above, the NCLB required states to establish accountability systems based on annual assessments. However, many states lowered their standards to avoid penalties, prioritizing superficial compliance over substantive improvement.
\begin{quote}
``The federal government’s ambitious goals in NCLB were not matched by sufficient attention to how teachers and administrators could realize these goals; there was a large disconnect between policy and practice” (Hills).
\end{quote}
This disconnect is part of a broader trend in federal education policy, where noncompliance or superficial adherence undermines the effectiveness of ambitious reforms. Programs like Race to the Top, which sought to incentivize innovation and accountability, similarly faced challenges in implementation, with many states adopting policies on paper to secure funding but failing to make substantive changes in practice. These patterns reveal a recurring misalignment between federal goals and local execution, raising critical questions about the feasibility of using federal mandates to drive systemic change in a decentralized education system.

\subsection{Inequitable Funding Mechanisms}
Furthermore, education federalism perpetuates funding disparities by tying school budgets to local property taxes. This reliance inherently links a district’s resources to its wealth, creating stark inequities between affluent and low-income areas. Affluent districts are able to fund better facilities, teacher salaries, and curricular programs, while poorer districts often struggle to meet basic needs.
\begin{quote}
  ``By relying largely on property taxes to fund schools... districts create funding gaps from the word go. Affluent areas end up with well-funded schools and low-income areas end up with poorly funded schools” (SOE Online).
\end{quote}
This funding disparity disproportionately affects students of color. In fact, districts serving the greatest concentrations of Black, Latino, or Native American students receive, on average, \$1,800 less per student than those serving fewer students of color (Harvard). While federal programs like Title I attempt to address these gaps, their impact is often undermined by structural loopholes and insufficient allocations. For instance, the "supplement not supplant" requirement—meant to ensure federal funds add to, rather than replace, local funding—has been inconsistently enforced, allowing wealthier districts to benefit disproportionately.\\
\\
These inequities are compounded by the lack of a unified national standard for funding adequacy (Learning Policy Institute). States operate with varying levels of commitment to educational equity, and in some states, school finance reforms have stagnated or regressed, leaving underfunded districts without the resources needed to support student success.

\subsection{Tension Between Local Autonomy and National Goals}
As seen with NCLB and ESSA, federalism often amplifies ideological divides, particularly in the context of education reform. State and local governments prioritize autonomy, frequently resisting federal mandates designed to promote equity. And while the initiatives mentioned above sought to establish accountability systems, states often manipulated standards to avoid federal consequences, undermining their effectiveness.
\begin{quote}
  ``Rather than propose rigorous common standards for all students, the existing structure of education federalism led Congress to allow each state to set its own academic standards. Instead, many states used the flexibility within NCLB as an opportunity to lower standards to those that were easily attainable” (Bowman).
\end{quote}
This resistance to federal oversight reflects broader political and cultural tensions. Wealthier states and districts, which often have greater fiscal capacity, oppose redistributive policies that would direct resources to underfunded areas. Meanwhile, local communities fear the loss of control over curriculum and school operations, perceiving federal mandates as intrusive and disconnected from local needs (Robinson). This dynamic creates a fragmented system where the quality of education a student receives is heavily influenced by geographic location.
\begin{quote}
``The reality of local control of education for many communities means the ability to control inadequate resources that provide many students substandard educational opportunities'' (Hills).
\end{quote}
This not only undermines efforts to close achievement gaps but also perpetuates a cycle of superficial compliance. As a result, the very principles of equity and adequacy that federal programs aim to uphold are often sidelined by competing priorities at the state and local levels.








\section{Opportunities for Reform and Innovation}
As shown above, the current education federalism model, while rooted in the intent to address the needs of diverse communities, has struggled to ensure equity and consistency across states. However, opportunities for reform and innovation exist to address inefficiencies in funding, foster collaboration across levels of government, and promote equity in educational outcomes. This section explores these opportunities by examining funding reforms, collaborative federalism, and equity-focused policies.


\subsection{Funding Reforms}
One of the most persistent challenges in education federalism is the reliance on local property taxes, which perpetuates disparities in funding across districts. 
Wealthier districts generate more revenue, leading to inequitable access to resources, while poorer districts struggle to provide basic educational opportunities. 
Federal reforms must address this imbalance to ensure more equitable resource distribution.\\
\\
Revising Title I of the Elementary and Secondary Education Act (ESEA) presents a critical opportunity. Title I currently provides federal funds to schools serving low-income students but is plagued by inefficiencies in targeting. As Congress recognized early on, Title I funds must supplement, not replace, local funding. However, “statutory loopholes—such as teacher salary, funding equivalences, and waivers—often thwart its effectiveness” (Harvard). Closing these loopholes and adopting stricter enforcement mechanisms for state compliance could enhance Title I's impact on resource equity.\\
\\
Federal matching grants also hold promise as an incentive for states to adopt equitable funding models. By tying federal aid to state efforts to close funding gaps, such grants can encourage states to prioritize underserved districts while not violating local and state autonomy. Furthermore, alternative models like federal-state education trusts could provide a stable funding base, reducing dependence on volatile local revenues.
However, funding reform is inherently complex, and further research is needed to understand how proposed changes will affect state compliance, resource distribution, and long-term equity before implementing new policies.



\subsection{Federalist Policy Innovations}
Collaborative federalism, which blends federal oversight with state and local innovation, offers a model for addressing education challenges without imposing one-size-fits-all solutions. This approach emphasizes partnerships, where the federal government acts as both a guide and a supporter of state-led initiatives. 
\begin{quote}
  ``The federal government can serve as a laboratory assistant, simultaneously observing and providing research support for a number of experiments, drawing inferences from the results and serving as an information repository” (McGovern).
\end{quote}
Programs like Race to the Top demonstrate the potential of such partnerships. By incentivizing states to implement innovative policies through competitive grants, the program encouraged experimentation in areas such as teacher evaluations and school turnarounds (McGovern). However, critics noted that the federal government’s heavy-handed selection criteria limited the range of reforms. Moving forward, federal initiatives should focus on unbiased assessments of proposals, supporting diverse strategies to foster educational innovation.\\
\\
A key component of collaborative federalism is the establishment of national baseline standards for funding or outcomes, coupled with state flexibility in achieving these benchmarks. Specifically, federal requirements should emphasize ``authentic measures of relevant knowledge and skills” while allowing states to design implementation strategies tailored to local needs (Learning Policy Institute). For example, pilot programs could explore varying teacher compensation models, such as bonuses for student progress or salary adjustments based on performance metrics, providing insights into effective incentives for student achievement.\\
\\
However, it is important to keep in mind that the success of federalist reforms hinges on fostering genuine cooperation between state and federal governments. By combining national vision with local adaptability, these reforms can leverage the strengths of both levels of governance to address disparities and drive meaningful educational progress.
\subsection{Addressing Systemic Disparities}
When talking about theeoretical federalism reform, the idea of equity must remain a central focus of legislative efforts---an idea spearheaded by the federal government. As one report notes, ``education federalism has failed to produce widespread educational quality [...] A greater federal role is necessary” (Bowman).
\\
\\
Expanding federal oversight in targeted areas can help address persistent disparities in access to high-quality education. One avenue for promoting equity is the development of common Opportunity-to-Learn (OTL) standards. These standards would identify the minimum resources—such as qualified teachers, technology, and adequate facilities—that states must provide to ensure educational equity. With these efforts and more collaboration, federal oversight, combined with technical assistance to states, could help close resource gaps and promote continuous improvement.\\
\\
Another critical area for reform is the development and implementation of culturally responsive policies in education. Federal programs that support bilingual education and prioritize local cultural needs play a vital role in bridging the achievement gap for historically underserved communities (Learning Policy Institute). For instance, initiatives that integrate language preservation with academic instruction can significantly improve educational outcomes for non-native English speakers (McGovern). Especially with the increasing influx of immigration and the growing cultural diversity in schools, implementing such policies has become an urgent priority to ensure that all students have access to equitable and effective education.
\subsection{Federalism as a Governance Model}
The lessons of education federalism extend beyond schools to other policy areas, such as healthcare and infrastructure. In each of these domains, federalism offers a framework for blending national standards with local flexibility, fostering diverse solutions to complex challenges---allowing these frameworks to ``spillover'' into other policy areas. For example, healthcare policies could benefit from a cooperative federalism approach that allows states to innovate while ensuring equitable access to care across the nation (Shanor).\\
\\
In particular, education federalism also highlights the potential of federalism to foster both diversity and unity in policy outcomes. By empowering states to serve as laboratories of innovation while upholding national equity standards, federalism can address local needs without sacrificing broader goals. However, realizing this potential requires a commitment to shared accountability and a recognition of the federal government’s role in protecting vulnerable populations. 
\section{Conclusion}
Federalism has long shaped the governance of public education in the United States, offering both opportunities and challenges in balancing national priorities with local autonomy. While this decentralized model fosters innovation and responsiveness to local needs, it has also entrenched disparities in funding, resources, and outcomes, particularly for historically underserved communities. These challenges underscore the need for reform to create a system that ensures equity while leveraging the strengths of federalism.  \\
\\
The analysis of programs like No Child Left Behind and the Every Student Succeeds Act demonstrates the difficulty of aligning federal oversight with state and local implementation. While federal mandates can highlight inequities and drive national standards, they often falter without sufficient support and collaboration. Conversely, excessive local autonomy risks perpetuating systemic disparities that undermine the ideal of equal opportunity.\\
\\
Moving forward, an effective education governance framework must strike a nuanced balance between national accountability and local flexibility. Federal efforts should focus on equitable resource distribution, robust Opportunity-to-Learn standards, and fostering collaborative partnerships with states and districts. These reforms, coupled with innovative approaches to addressing disparities, can transform federalism into a tool for progress rather than a source of division.\\
\\
Ultimately, federalism’s success in education depends on its ability to bridge ideological and structural divides, creating a governance model that reflects both the diversity and shared values of the nation. By prioritizing equity and collaboration, policymakers can ensure that the promise of education as a cornerstone of opportunity is fulfilled for all students, regardless of where they live or their circumstances.


\pagebreak
\section{Sources}
\begin{hangparas}{.25in}{1}
Wong, K.K. (2018) Education, Center for the Study of Federalism. \url{https://federalism.org/encyclopedia/no-topic/education/} (Accessed: 25 November 2024).
\end{hangparas}
\par \vspace{0.5cm}
\begin{hangparas}{.25in}{1}
Hornbeck, D. (2017) Federal role in education has a long history, The Conversation. \url{https://theconversation.com/federal-role-in-education-has-a-long-history-74807} (Accessed: 25 November 2024). 
\end{hangparas}
\par \vspace{0.5cm}
\begin{hangparas}{.25in}{1}
Pelsue, B. (2017) When it comes to education, the federal government is in charge of ... um, what?, Harvard Graduate School of Education. \url{https://www.gse.harvard.edu/ideas/ed-magazine/17/08/when-it-comes-education-federal-government-charge-um-what} (Accessed: 25 November 2024). 
\end{hangparas}
\par \vspace{0.5cm}
\begin{hangparas}{.25in}{1}
Kosar, K. (2015) Kill the Department of Ed.? It’s been done, The Agenda. \url{https://www.politico.com/agenda/story/2015/09/department-of-education-history-000235} (Accessed: 25 November 2024). 
\end{hangparas}
\par \vspace{0.5cm}
\begin{hangparas}{.25in}{1}
Walsh, A., Moynihan, M. and Yin, E. (2022) Has the ‘Every student succeeds act’ left children behind?, The Regulatory Review. \url{https://www.theregreview.org/2022/08/06/saturday-seminar-has} \url{-the-every-student-succeeds-act-left-children-behind/} (Accessed: 25 November 2024). 
\end{hangparas}
\par \vspace{0.5cm}
\begin{hangparas}{.25in}{1}
Burnes, D.W. (1978) ‘A Case Study of Federal Involvement in Education’, Proceedings of the Academy of Political Science, 33(2), pp. 87–98. doi:10.2307/1173920. 
\end{hangparas}
\par \vspace{0.5cm}
\begin{hangparas}{.25in}{1}
McGuinn, P. (2018) No Child Left Behind Act, Center for the Study of Federalism. \url{https://encyclopedia.federalism.org/index.php/No-Child-Left-Behind-Act} (Accessed: 25 November 2024).
\end{hangparas}
\par \vspace{0.5cm}
\begin{hangparas}{.25in}{1}
Sunderman, G. and Kim, J. (2004) ‘Expansion of Federal Power in American Education: Federal-State Relationships Under the No Child Left Behind Act, Year One ’, The Civil Rights Project: Harvard University [Preprint]. \url{https://civilrightsproject.ucla.edu/research/k-12-education/nclb-title-i/} \url{expansion-of-federal-power-in-american-education-federal-state-relationships-under-the-no-child} \url{-left-behind-act-year-one/sunderman-kim-expansion-federal-power-american.pdf}. 
\end{hangparas}
\par \vspace{0.5cm}
\begin{hangparas}{.25in}{1}
Dee, T.S. and Jacob, B.A. (2010) ‘The Impact of No Child Left Behind on Students, Teachers, and Schools’, Brookings Papers on Economic Activity, 2010(2), pp. 149–194. doi:10.1353/eca.2010.0014. 
\end{hangparas}
\par \vspace{0.5cm}
\begin{hangparas}{.25in}{1}
Heise, M. (2017) ‘From No Child Left Behind to Every Student Succeeds: Back to a Future for Education Federalism’, Columbia Law Review, 117(7), pp. 1859–1896. doi:10.31228/osf.io/kdfje. 
\end{hangparas}
\par \vspace{0.5cm}
\begin{hangparas}{.25in}{1}
Sundquist, C. (2012) ‘Positive Education Federalism: The Promise of Equality after the Every Student Succeeds Act’, Mercer Law Review, 351(68). doi:10.2139/ssrn.2169455. 
\end{hangparas}
\par \vspace{0.5cm}
\begin{hangparas}{.25in}{1}
Saultz, A., Fusarelli, L.D. and McEachin, A. (2017) ‘The Every Student Succeeds Act, the Decline of the Federal Role in Education Policy, and the Curbing of Executive Authority’, Publius: The Journal of Federalism, 47(3), pp. 426–444. doi:10.1093/publius/pjx031. 
\end{hangparas}
\par \vspace{0.5cm}
\begin{hangparas}{.25in}{1}
Pears, E. and Sydnor, E. (2022) ‘COVID-19 and the Culture of American Federalism’, RSF: The Russell Sage Foundation Journal of the Social Sciences, 8(8), pp. 181–220. doi:10.7758/rsf.2022.8.8.09. 
\end{hangparas}
\par \vspace{0.5cm}
\begin{hangparas}{.25in}{1}
García, E. and Weiss, E. (2020) ‘COVID-19 and Student Performance, Equity, and U.S. Education Policy’, Economic Policy Institute [Preprint]. \url{https://epi.org/205622}. 
\end{hangparas}
\par \vspace{0.5cm}
\begin{hangparas}{.25in}{1}
Bowman, K.L. (2017) ‘The Failure of Education Federalism ’, University of Michigan Journal of Law Reform, 51. doi:10.2139/ssrn.2876889. 
\end{hangparas}
\par \vspace{0.5cm}
\begin{hangparas}{.25in}{1}
Hills, R.M. (2012) ‘The Case for Educational Federalism: Protecting Educational Policy from the National Government’s Diseconomies of Scale’, Notre Dame Law Review, 87(5). \url{https://scholarship.law.nd.edu/ndlr/vol87/iss5/6}. 
\end{hangparas}
\par \vspace{0.5cm}
\begin{hangparas}{.25in}{1}
Robinson, K.J. (2013) ‘The High Cost of Education Federalism’, Wake Forest Law Review, 287(48). \url{https://papers.ssrn.com/sol3/papers.cfm?abstract-id=2259631}. 
\end{hangparas}
\par \vspace{0.5cm}
\begin{hangparas}{.25in}{1}
Harvard, C.J. (2009) ‘Funny Money: How Federal Education Funding Hurts Poor and Minority Students’, University of Baltimore Law Review [Preprint]. \url{https://scholarworks.law.ubalt.edu/cgi/viewcontent.cgi?article=1306}. 
\end{hangparas}
\par \vspace{0.5cm}
\begin{hangparas}{.25in}{1}
SOE Online (2022) Inequality in Public School Funding: Key Issues \& Solutions for Closing the Gap, American University: School of Education. \url{https://soeonline.american.edu/blog/inequality-in-public-school-funding/} (Accessed: 25 November 2024). 
\end{hangparas}
\par \vspace{0.5cm}
\begin{hangparas}{.25in}{1}
Learning Policy Institute. (2020). The Federal Role in Advancing Education Equity and Excellence. Palo Alto, CA: Author. \url{https://learningpolicyinstitute.org/product/advancing-education-2020-brief} (Accessed: 25 November 2024)
\end{hangparas}
\par \vspace{0.5cm}
\begin{hangparas}{.25in}{1}
McGovern, S.K. (2018) ‘A New Model for States as Laboratories for Reform: How Federalism Informs Education Policy’, NYU Law Review, 86(5). \url{https://nyulawreview.org/wp-content/uploads/2018/08/NYULawReview-86-5-McGovern.pdf}.
\end{hangparas}
\par \vspace{0.5cm}
\begin{hangparas}{.25in}{1}
Shanor, C. (2015) Spillover Federalism, JOTWELL. \url{https://conlaw.jotwell.com/spillover-federalism/} (Accessed: 25 November 2024)
\end{hangparas}


\end{document}
